\nnchapter{ЗАКЛЮЧЕНИЕ}

В данной работе была поставлена и достигнута цель, связанная с разработкой системы построения программной траектории
и регулятора движения по ней беспилотного наземного транспортного средства.

В ходе работы были рассмотрены основные методы планирования движения беспилотных автомобилей и других автономных
наземных транспортных средств. Проведен анализ способов методов планирования движения и выделен наиболее подходящий
для использования в разработке системы управления.

На основе данных анализа литературных источников был разработан алгоритм построения программной траектории для
движения к заданной цеди и объездом препятствий и регулятор для движения по программной траектории с обратной связью.

Используя предложенный подход была спроектирована и реализована система управления движением беспилотного транспортного
средства. Была собрана мобильная колесная платформа, оснащенная датчиками  системы компьютерного зрения
(стереокамерой и LiDAR) с целью экспериментальной проверки разработанных алгоритмов. Была разработана подсистема
компьютерного зрения, осуществляющая обнаружение препятствий с использованием облаков точек.

Было проведено экспериментальное исследование работы системы построения программной траектории и регулятора движения
по ней беспилотного транспортного средства, которое показало работоспособность разработанных решений.

Научная новизна работы заключается в следующих положениях:
\begin{enumerate}
    \item был разработан алгоритм построения программной траектории для наземного беспилотного транспортного
          средства, осуществляющий построение траектории в форме полиномов пятого порядка, путем составления
          графа возможных траекторий и нахождения оптимальной с помощью алгоритма Дейкстры;
    \item был разработан регулятор для движения по программной траектории с использованием обратной связи
          от системы SLAM.
\end{enumerate}

Дальнейшее улучшение возможностей системы построения программных траекторий может быть осуществлено путем использования
программы "ФРУНД" для проверки и/или формирования участков траектории с целью достижения лучшего учета динамических
ограничений транспортного средства.