\nnchapter{ЗАКЛЮЧЕНИЕ}

В данной работе была поставлена и достигнута цель, связанная с реализацией системы управления роботом-гексаподом.

В ходе работы были рассмотрены технические средства, необходимые для создания данной системы, основные парадигмы проектирования систем управления роботами. Проведен анализ и разбор аналогичного решения с выделением положительных и отрицательных моментов и путей усовершенствования разрабатываемой системы.

На основе изучения существующих систем управления, был выбран необходимый подход для управления инсектоморфным роботом, с использованием системы ФРУНД в качестве генератора траекторий движения.

Используя данный подход был реализован кросплатформенный адаптивный web интерфейс управления, модули осуществляющие связь параметров, получаемых от аппаратного и виртуального пультов, системы ФРУНД и модулей управления виртуальным и аппаратным роботом-гексаподом.

Были проведены эксперименты по тестированию решения, полученного в ходе выпускной бакалаврской работы, которое по результатом проявило полное соответствие поставленной задаче.