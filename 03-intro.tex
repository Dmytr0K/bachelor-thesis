\nnchapter{ВВЕДЕНИЕ}
\aftertitle

В настоящее время активно ведутся исследования и разработки в области создания и использования инсектоморфных роботов, а также реализация систем управления и ориентирования роботов данного типа в пространстве. Данные тенденции обусловленны тем, что этот вид роботов является более проходимым, по сравнению с другими, которые, как правило, оборудуются колесными и гусеничными движителями.

Как можно заметить, рельеф земли, в основном, представляет собой неровные поверхности, соответственно можно выделить следующие преимущества инсектоморфных роботов: 
\begin{itemize}
	\item возможность самостоятельно выбирать точку касания с поверхностью;
	\item адаптивное движение по отношению к поверхности.
\end{itemize}

Стоит отметить, что не каждого шагающего робота можно считать "настоящим"{} роботом, так как в наше время создано  множество различных моделей, которые шагают сугубо номинально, используя лишь циклические походки и циклическую генерацию движения, однако этого недостаточно. Для настоящей ходьбы робот должен быть оснащен различными инерционными датчиками, иметь возможность измерения неровности поверхности по которой производится движение, расстояние до нее, а также уметь подстраивать свою походку под эти неровности, обеспечивая возможность регулируемого взаимодействия с опорной поверхностью.

Такое стало возможным только в последнее время с развитием вычислительной техники. Этому способствовало появление компьютеров, обладающих достаточными вычислительными мощностями и ресурсами и при всем при том в сравнительно компактном корпусе, что позволяет устанавливать их непосредственно на робота и в режиме реального времени обрабатывать показания, которые принимаются с датчиков и за достаточно маленький период времени изменять положение приводов, основываясь на этих показаниях.

Поэтому актуальной является работа по разработке системы для решения задачи удаленного управления инсектоморфным роботом в режиме реального времени с использованием генерации движений за помощью математического пакета ФРУНД.


Целью данной выпускной бакалаврской работы является разработка ПО для обеспечения возможности передачи параметров, необходимых для генерации движения непосредственно во ФРУНД, а также интегрирование разработанного ПО, как часть системы управления инсектоморфным роботом.

Для достижения данной цели необходимо решить следующие задачи:

\begin{enumerate}
    \item анализ предметной области существующих способов управления роботами такого типа;
    \item постановка задачи проектирования подстистемы формирования параметров генерации движения;
    \item реализация подсистемы формирования параметров управления и её интеграция в существующей системе управления роботом;
    \item проведение экспериментов и оценка результатов работы.
\end{enumerate}

В первой главе настоящей работы приведен обзор различных видов роботов и методов управления ими.

Во второй главе рассматривается проектирование системы управления роботом-гексаподом, сравнение с аналогом и постановка требований к реализации.

В третьей главе рассматривается этап реализации системы управлением роботом гексаподом и необходимые компоненты данной системы.

В четвертой главе проводятся тесты системы управления на соответствие поставленным требованиям.

В разделе "Выводы"{} подводятся итоги создания подсистемы дистанционного управления инсектоморфным роботом и рассматриваются результаты экспериментов.