% TODO: Длинные заголовки при переносе строки продолжаются под цифрами. Т.е. не начало текста не выровнено
% ----------------------------- БАЗОВЫЕ НАСТРОЙКИ  -----------------------------

\documentclass[oneside, final, 14pt,  a4paper]{extreport}    % Класс документа. Чаще всего использует его
\usepackage[T2A]{fontenc}
\usepackage[utf8]{inputenc}                        % Кодировка utf8x для linux
\usepackage[english,russian]{babel}                % Переносы и прочее для русского и английского
\usepackage{cmap}                                  % Нормальные русские символы в получаемом pdf
\usepackage[linktocpage=true]{hyperref}            % Гиперссылки
\usepackage{indentfirst}                           % Красная строка для первых абзацев
\usepackage{mathtools}
\usepackage{tabularx}                              % Много табличной боли и магии
\usepackage{color, colortbl}                       % Добавляем кольор
\definecolor{Gray}{gray}{0.9}                      % Добавляем серый цвет
\usepackage[pdftex]{graphicx}                      % Для вставки картинок
\usepackage{listings}                              % Для вставки кода
\usepackage{titlesec}                              % Для настройки названий глав, разделов итп
\usepackage{titletoc}                              % Для настройки оглавления
\usepackage{enumitem}                              % Для настройки списков
\usepackage{flafter}                               % Плавающие элементы встречаются после ссылки на них
\usepackage{caption}                               % Настройка подписей плавающих элементов
\usepackage{setspace}                              % Для настройки интервалов
\usepackage{color, soulutf8}                       % Выделенный текст
\usepackage{csquotes}
\usepackage{algorithm}                             % Псевдокод
\usepackage{algpseudocode}
\usepackage{chngcntr}                              % Чтоб настроить сквозную нумерацию
\usepackage{lastpage}                              % Получить количество страниц
\usepackage[figure,table]{totalcount}              % Получить количество рисунков и таблиц

%ссылки
\usepackage{xcolor}
\usepackage{hyperref}
\hypersetup{colorlinks,
  pdftitle={Diplom},
  pdfauthor={Dmytro Kukhno},
  allcolors=[RGB]{000 000 000}}
  
% Хак для того чтобы аннотация была на 4 странице
\setcounter{page}{0}


% Библиография
\usepackage[
    backend=biber,
    sorting=none,                                  % Сортировка в порядке цитирования
	style=gost-numeric,
	language=auto,                                 % Автовыбор стиля, напр. писать [et al] вместо [и др]
	autolang=other                                 % для англоязычных публикаций (langid={english})
] {biblatex}

\linespread{1.3}                                   % Полуторный интервал

% Настройка полей
\usepackage{geometry}
\geometry{left=3cm}
\geometry{right=1cm}
\geometry{top=1.5cm}
\geometry{bottom=2cm}


\sloppy                                           % Избегать залезания строк на поля (надо?)
\setlength\parindent{1.5cm}                       % Отступ красной строки

\newcommand{\nf}{\normalfont}

% Для tabularx:
\newcolumntype{Y}{>{\centering\arraybackslash}X} % Растянутый столбец (как X) с выравниванием по центру
\newcolumntype{P}{>{\raggedleft\arraybackslash}X}% Растянутый столбец (как X) с выравниванием по справа

%--------------------------- НАСТРОЙКА ТИТУЛЬНОГО ЛИСТА ------------------------

% Текст без межстрочного интервала
\newenvironment{nospasing}
{
    \begin{spacing}{1}
}
{
    \end{spacing}
}

% Поля титульного листа с подписями и линями (подпись, ФИО итп)

% подпись под пустой линией заданной длины
\newcommand{\efield}[2][2cm]{
	%1 - ширина поля
	%2 - подпись под линией
	$\underset{\text{(#2)}}{\underline{\hspace*{#1}}}$
}

% Подпись под текстом
\newcommand{\tfield}[2]{
	%1 - содержимое поля
	%2 - подпись под линией
	$\underset{\text{(#2)}}{\underline{\smash{\text{#1}}}}$
}

% Название поля, линия до конца строки, значение поля над линией, подпись под линией

\newcommand{\lfield}[3]{
	%1 - название поля
	%2 - содержимое поля
	%3 - подпись под линией
	\noindent
	\renewcommand{\arraystretch}{0.7}
	\begin{tabularx}{\linewidth}{@{}lY@{}}
    	#1 & #2 \\
    	\cline{2-2}
           & \footnotesize(#3)\normalsize
	\end{tabularx}

}


% ----------------------------- НАСТРОЙКИ ЗАГЛАВИЙ -----------------------------
%% Отступ 1.5 слева (как у красной строки)t
% Нет точки между номером и названием
% Интервал между подзаголовками 1.5
% Интервал между заголовком и текстом 2*1.5
% Поддержка приложений

% Глава
\titleformat{\chapter}
	[block]                % Shape. block убирает перенос заглвания на новую строку
    {\normalfont}          % Format. Собственно, стиль
    {\thechapter}          % Label. Номер главы.
    {8pt}                  % Sep. Пробел между номером и главой (TODO: уточнить)
    {}                     % before-code. Код перед названием
\titlespacing*{\chapter}
	{1.5cm}                % Левый отступ (как у красной строки)
	{18pt}                 % Верхний отступ, 1.5 интервал
	{18pt}                 % Нижний отступ, 1.5 интервал

% Раздел
\titleformat{\section}
	{\normalfont}
	{\thesection}
	{8pt}{}
\titlespacing*{\section}
	{1.5cm}{18pt}{18pt}

% Подраздел
\titleformat{\subsection}
	{\normalfont}
	{\thesubsection}
	{8pt}{}
\titlespacing*{\subsection}
	{\parindent}{18pt}{18pt}

% Глава без номера (введение, заключение и т.п.)
\newcommand{\nnchapter}[1]
{
	\chapter*{#1}
	\addcontentsline{toc}{chapter}{#1}
}

% Фейковая глава для автореферата
% Зачем платить больше, если не нужно содержание?
\newcommand{\referchapter}[1]
{

    \vspace{18pt}
    #1
    \vspace{18pt}

}


% Приложения
% Использовать \chapter{} для создания приложений
% Очень грязный хак, но работает
\newcommand{\StartAppendix}
{
	\setcounter{chapter}{0}
}

\renewcommand{\appendix}[1]
{
	\newpage
	\stepcounter{chapter}
	\newcommand{\theappendix}{ПРИЛОЖЕНИЕ \MakeUppercase{\asbuk{chapter}}}
	\addcontentsline{toc}{chapter}{\texorpdfstring{\theappendix} ~--- #1}
	\begin{center}
		\theappendix\\
		{#1}
	\end{center}
}

% Расстояние между заглавиями и текстом должно быть 2 полуторных интервала,
% а расстояние между заглавиями - один полуторный интервал.
% Не придумал ничего лучше, кроме как вставлять вручную
\newcommand{\aftertitle}{\vskip 18pt}

% ----------------------------- НАСТРОЙКИ СОДЕРЖАНИЯ ---------------------------
% Нет выделения жирным
% Все с одним уровнем отступа
% Поддержка приложений

% Главы
\titlecontents{chapter}
	[0em] {}
	{\thecontentslabel~}{}
	{\titlerule*[1pc]{.}\contentspage}

% Разделы
\titlecontents{section}
	[0em] {}
	{\thecontentslabel~}{}
	{\titlerule*[1pc]{.}\contentspage}

% Подразделы
\titlecontents{subsection}
	[0em] {}
	{\thecontentslabel~}{}
	{\titlerule*[1pc]{.}\contentspage}

% Заголовок
\addto\captionsrussian{
	\renewcommand{\contentsname} {СОДЕРЖАНИЕ}
}

%-------------------------------- НАСТРОЙКИ СПИСКОВ ----------------------------
% Маркерный список
\setlist[itemize]{
	label=–,                  % Дефис в каяестве маркера
	leftmargin=1.5cm,         % Текст в списке выравнен по красной строке
	itemindent=15pt,          % Маркер выравнен по красной строке, т.е. первая строка чуть сдвинута на размер маркера
	nosep                     % Убираем интервал между пунктами списков
}

% Числовой
\setlist[enumerate]{
    label*=\arabic*),
    leftmargin=1.5cm,
    itemindent=20pt,
    nosep
}

%--------------------------- НАСТРОЙКИ РИСУНКОВ И ТАБЛИЦ -----------------------
% Рисунки подписываются "Рисунок N - ..." по центру
% Таблицы подписываются "Таблица N - ..." с левого края

\captionsetup[figure]{
	name=Рисунок,
	labelsep=endash,
	justification=centering,
	belowskip=-10pt,aboveskip=0pt,
	font={stretch=1.3}
}
\captionsetup[table]{name=Таблица, labelsep=endash, justification=raggedleft, singlelinecheck=false}

% Сквозная нумерация таблиц, рисунков и формул
\counterwithout{figure}{chapter}
\counterwithout{table}{chapter}
\counterwithout{equation}{chapter}
\pdfimageresolution=150

% Заголовок таблицы серый, выравнивание по центру
\makeatletter
\newcommand{\tabheader}[1]{\hline\rowcolor{Gray}\multicolumn{1}{|c|}{#1}\checknextarg}
\newcommand{\checknextarg}{\@ifnextchar\bgroup{\gobblenextarg}{\\\hline}}
\newcommand{\gobblenextarg}[1]{&\multicolumn{1}{c|}{#1}\@ifnextchar\bgroup{\gobblenextarg}{\\\hline}}

%\setlength{\belowcaptionskip}{-14pt}

%---------------------------------- ФОРМУЛЫ ------------------------------------

\newcommand{\degsym}{^{\circ}}    % Градус
\newcommand{\CST}{\mathcal{C}}    % C-State, пространство конфигурации
\newcommand{\XST}{\mathcal{X}}    % X-State, пространство состояний
\newcommand{\vect}[1]{\overrightarrow{#1}}
\DeclarePairedDelimiter\floor{\lfloor}{\rfloor}

%-------------------------------- БИБЛИОГРАФИЯ ---------------------------------



%\addbibresource{hexapod.bib}