\nnchapter{СПИСОК ИСПОЛЬЗОВАННОЙ ЛИТЕРАТУРЫ}
\aftertitle

\begin{enumerate}[label={\arabic*.},labelsep=2.4em,itemindent=0pt,leftmargin=*,align=right]
	\item Синтез локомоции шагания антропоморфного робота. / А.С. Горобцов [и др.] // Робототехника и искусственный интеллект: материалы VII Всероссийской научно-технической конференции с международным участием (г. Железногорск, 11 декабря 2015 г.) / под науч. ред. В.А. Углева. – Красноярск: Сиб. федер. ун-т, 2016. – С. 8-12.
	\item Дистанционно управляемые подрывные машины семейства Sd.Kfz.302/303 Goliath (Германия) [Электронный ресурс] // Военное обозрение. – Режим доступа: https://topwar.ru/94692-distancionno-upravlyaemye-podryvnye-mashiny-semeystva-sdkfz302-303-goliath-germaniya.html (дата обращ. 16.03.2020).
	\item Фокин, В. Г. Обзор и перспективы развития мобильных шагающих робототехнических систем / В. Г. Фокин, С. В. Шаныгин // Молодой ученый. – 2015. – №18. – С. 207-215.
	\item Autonomous Locomotion and Navigation of Anthropomorphic Robot / А.К. Титов, А.Е. Марков, А.В. Скориков, П.С. Тарасов, А.Е. Андреев, С.С. Алексеев, А.С. Горобцов // Creativity in Intelligent Technologies and Data Science. Second Conference, CIT\&DS 2017 (Volgograd, Russia, September 12-14, 2017): Proceedings / ed. by A. Kravets, M. Shcherbakov, M. Kultsova, Peter Groumpos ; Volgograd State Technical University [et al.]. – [Germany]: Springer International Publishing AG, 2017. – P. 242-255. – (Ser. Communications in Computer and Information Science; Vol. 754).
	\item GAZEBO [Электронный ресурс]: офиц. сайт. – Режим доступа: http://gazebosim.org (дата обращ. 02.04.20).
	\item ROS: an open-source robot operating system / M. Quigley, K. Conley, B. Gerkey et al // IEEE ICRA Workshop on Open Source Software – 2009. - Vol. 3.
	\item Черноножкин, В. А. Система локальной навигации для наземных мобильных роботов / В. А. Черноножкин // Научно-технический вестник информационных технологий, механики и оптики. – 2008. – №57. – С. 13-22.
	\item Инерциальная навигационная система [Электронный ресурс] // Системы навигации и позиционирования. – Режим доступа: https://glonass-std.ru/sistemy-navigatsii/inertsialnaya-navigatsionnaya-sistema.html (дата обращ. 12.04.2020).
	\item ФРУНД [Электронный ресурс]: офиц. сайт. – Режим доступа: http://frund.vstu.ru (дата обращ. 13.04.20).
	\item Ковальчук, А. К. Принципы построения программного обеспечения системы управления антропоморфным шагающим роботом. / А.К. Ковальчук, Д.Б. Кулаков, С.Е. Семенов // Автоматизация и современные технологии. – Москва, 2007. № 2.
	\item Vue.js the Progressive JavaScript Framework [Электронный ресурс]: офиц. сайт. – Режим доступа: https://vuejs.org (дата обращ. 15.04.20).
	\item ROSBridge\_suite – ROS Wiki [Электронный ресурс]: офиц. сайт. – Режим доступа: http://wiki.ros.org/rosbridge\_suite (дата обращ. 16.04.20).
	\item Brown, J. H. How fast is fast enough. Choosing between Xenomai and Linux for real-time applications / H. Brown, B. Martin. [Электронный ресурс] – 2010 – Режим доступа: https://www.osadl.org/fileadmin/dam/rtlws/12/Brown.pdf (дата обращ. 22.05.20).
	\item Edwards, S. ROS-industrial: applying the robot operating system (ROS) to industrial applications / S. Edwards, C. Lewis // IEEE Int. Conference on Robotics and Automation. - ECHORD Workshop. - St. Paul, 2012.
	\item Development of the insectoid walking robot with inertial navigation system / В.А. Егунов, А.Л. Качалов, М.К. Петросян, П.С. Тарасов, Е.В. Янкина // Proceedings of the 2018 International Conference on Artificial Life and Robotics (ICAROB2018) (February 1-4, 2018, B-CON Plaza, Beppu, Oita, Japan) / Editor-in- Chief Masanori Sugisaka ; International Steering Committee of International Conference on Artifical Life and Robotics (ICAROB), ICAROB society (ALife Robotics Corporations Ltd.), IEEE Fukuoka Section (Japan). – [Japan], 2018. – P. 54 (Mobile Robotics ; OS7-2).
	\item The Walk-Man Robot Software Architecture [Электронный ресурс] // ResearchGate. – Режим доступа: https://www.researchgate.net/publication/302916141\_The\_Walk-Man\_Robot\_Software\_Architecture (дата обращ. 23.05.20).
 	\item Hayes-Roth, B. A blackboard architecture for control / B. Hayes-Roth // Artif. Intell. – 1985 (26). – P. 251–321. DOI: 10.1016/0004-3702(85)90063-3.
 	\item The Ach library: a new framework for real-time communication / N. Dantam, D. Lofaro, A. Hereid et al. // IEEE Robot. Autom. – 2014, Mag. 22, P. 76–85. DOI: 10.1109/MRA.2014.2356937.
 	\item Orebäc, A. Evaluation of architectures for mobile robotics / A. Orebäck, H. I. Christensen // Robots – 2003. – (14). – P. 33–49. DOI: 10.1023/A: 1020975419546.
 	\item Magyar, G. Comparison study of robotic middleware for robotic applications / G. Magyar, P. Sinčák, Z. Krizsan // Emergent Trends in Robotics and 101 Intelligent Systems (Cham: Springer International Publishing AG). – 2015. – Vol. 316. – P. 121–128.
\end{enumerate}