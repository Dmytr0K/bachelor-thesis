\nnchapter{ВВЕДЕНИЕ}
\aftertitle

В настоящее время активно ведутся исследования и разработки в области беспилотных транспортных средств, 
в частности, беспилотных автомобилей. Бесплотные транспортные средства являются очень перспективной
технологией, на которую делают ставку многие крупные автопроизводители и IT-компании. В частности, 
исследованиями и разработками в этой области занимаются такие компании как Waymo (Google), Tesla, Audi \cite{company_1},
Яндекс \cite{company_yandex}, GM, Apple, Uber, Intel совместно с BMW и ряд других.
Наибольших успехов в этом достигли Waymo, которая запустила тестовый коммерческий сервис беспилотного
такси для ограниченного круга испытателей в нескольких городах штата Аризона, Яндекс, запустивший беспилотное 
такси в Иннополисе и технопарке Сколково, и компания Tesla, выпускающая серийные автомобили с ограниченными
функциями автопилота. \hl{добавить источники}

Массовое применение беспилотных автомобилей на дорогах общего пользования может привести к ряду положительных изменений, таких как
\begin{itemize}
    \item уменьшение человеческого фактора, что уменьшит количество ДТП и увеличит безопасность на дорогах;
    \item увеличению пропускной способности дорог и уменьшение пробок, потому что беспилотные автомобили
          смогут поддерживать меньшее расстояние друг с другом, быстрее реагировать на изменения дорожной
          ситуации;
    \item увеличение плотности парковок и иных инфраструктурных объектов по тем же причинам, что позволит
          сэкономить площадь;
    \item уменьшение количества личных автомобилей, по причине того, что, согласно анализу, один общественный
          беспилотный автомобиль может заменить девять--тринадцать личных автомобилей без компромиссов в 
          текущих сценариях использования \cite{overview_private_ovnership}, что еще больше сократит нагрузку
          на инфраструктуру, уменьшит пробки и сократит вредные выбросы.
\end{itemize}

Эти и ряд других возможностей делают беспилотные транспортные средства крайне полезными и перспективными в 
будущем.

\hl{И вот как-то надо перейти к тому, какого хера мы это делаем

Поэтому актуальной задачей является сделать вид, что мы тоже не лохи и изобразить какое-то подобие 
беспилотного автомобиля, чтобы потом еще десять лет показывать на каких-нибудь выставках и днях открытых
дверей.}


Целью данной диссертационной работы является разработка и реализация методов управление движением беспилотного
автомобиля. Для достижения данной цели необходимо решить следующие задачи:

\begin{enumerate}
    \item анализ существующих подходов к задаче управления беспилотными автомобилями в целом, выделение 
          типичных подсистем,
    \item проектирование подсистем управления движением беспилотным автомобилем,
    \item реализация подсистем управления движением,
    \item проведение экспериментов и оценка результатов работы. 
\end{enumerate}

В первой главе настоящей работы приведен обзор и анализ существующих подходов к построению систем управления
беспилотными автомобилями в целом, анализ возможных структур систем управления, выделение типичных подсистем.

Во второй главе рассматривается проектирование подсистем, отвечающих за управление движением автономного
автомобиля. 

В третьей главе рассматривается реализация рассмотренных подсистем, а также методика их применения для управления
беспилотным автомобилем.  

В четвертой главе рассматриваются экспериментальная часть работы, связанная с использования разработанной системы
для проведения исследования движений автомобиля. 

В разделе "Выводы" подводятся итоги создания подсистемы управления беспилотным автомобилем и рассматриваются
результаты экспериментов.

\hl{Научная новизна и актуальность. Хотя, актуальность вроде расписал, но не сказал явно это слово, надо сказать}
